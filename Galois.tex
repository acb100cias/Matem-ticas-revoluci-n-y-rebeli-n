\chapter{Galois Revolución o muerte}

A penas anocheciendo, un joven de veintitantos años escribe de forma frenética en su habitación, una habitación pequeña, de estudiante.  Pero el ya no es estudiante, ha estado dos veces en la cárcel y se encuentra en las listas negras de la policía política de la monarquía restaurada francesa. Sus preocupaciones no son de índole político, aunque en las últimas semanas este aspecto de su vida ha consumido todo su tiempo. 

En realidad lo que ocupa su mente es la  certeza de que la madrugada siguiente morirá, ha sido  retado a un duelo por un provocador infiltrado en el movimiento. Su muerte dirán los reportes policíacos, sera resultado de un lío de faldas, cuando en realidad se trata de una ejecución extrajudicial.

Sin embargo el trabajo que ahora lo afana es terminar una monografía que contiene los últimos resultados que ha descubierto estudiando un problema algebraico importante. ¿Pueden encontrarse soluciones para ecuaciones algebraicas de grado mayor que 5? Hacía algunos años el joven matemático Niels Heinrich Abel había respondido esa pregunta en general de manera negativa. 

 El joven Evariste, que esta noche escribe como energúmeno, con la certeza de su muerte pendiendo sobre su cabeza, ha estudiado durante varios años este problema y ha caracterizado los casos en los que es posible hallar las soluciones de estas ecuaciones, él no lo sabe, pro acaba de fundar una nueva rama de las matemáticas. 

El anterior parece un relato salido de una novela de Victor Hugo, en realidad es una historia real que, en parte, inspiró algunos personajes de la novela  los Miserables. El joven a quien se refiere el relato
