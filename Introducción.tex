\chapter{Introducción}

La historia humana es una gran enredadera, regada con la sangre de los pueblos y abonada con las vidas de un puñado de héroes, hay pocos que se atreven a negarlo. 


Alguna vez escuche a alguien argumentar que se puede ser un buen militante o un buen académico pero no ambos, sin embargo hay ejemplos en la historia de individuos que han podido al mismo tiempo ser militantes consecuentes y grandes académicos...

Desde aquellos matemáticos que acompañaron la construcción del régimen republicano en Francia, quienes como Evariste Galois se opusieron a los abusos cometidos por el régimen de la monarquía restaurada, pasando por los constructores de la escuela soviética de matmáticas o los socialistas que se opusieron al fascismo en Italia, España y Alemania. 

También están aquellos que realizaron valientes actos de espionaje durante la guerra fria y los que se opusieron vehementemente, en las aulas y las calles a las agresiones imperialistas de EU en la posguerra, aquellos que perdieron la vida luchando una guerra de guerrillas destinada al fracaso y desde luego a  los que desde su trinchera lucharon por la gratuidad de la educación en México y que hoy son parte del movimiento de la 4a Transformaciòn que gobierna nuestro país por un segundo periodo..

En efecto combinar ambas actividades puede parecer complicado pero la mayoría de las cosas que valen la pena en este mundo lo son.

